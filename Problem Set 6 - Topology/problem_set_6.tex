\documentclass[12pt]{article}
\usepackage[margin=1in]{geometry} 
\usepackage{amsmath}
\usepackage{amssymb}
\usepackage{amsthm}
\usepackage{accents}


\setlength{\oddsidemargin}{0in}
\setlength{\textwidth}{6.5in}
\setlength{\topmargin}{-.55in}
\setlength{\textheight}{9in}
\pagestyle{empty}
\renewcommand \d{\displaystyle}
\renewcommand \a{\shortstack{$\rightarrow$\\$u$}}
\renewcommand \b{\shortstack{$\rightarrow$\\$v$}}

\begin{document}
\noindent Math 4510

\noindent Topology

\vspace{.2in}
\begin{center}
Problem Set 6
\end{center}

 \begin{enumerate}%\setlength{\itemindent}{-1.5em}
 %\item (\#1\&\#2 in 4.4) Given the following subsets of $\mathbb{R}$:\\
%$A=\{1,2,3\}$\\
%$B=(0,1)\cap\mathbb{Q}$\\
%$C=[0,1]\cup(2,3]\cup\{4\}$
%begin{enumerate}
%\item Find the interior of each subset when considered as subsets of $\mathbb{R}_{\mathcal{U}}$.
%\item Find the interior of each subset when considered as subsets of $\mathbb{R}_{\mathcal{RR}}$.
%\end{enumerate}

\item Prove Theorem 4.4.3: For any subset $A$ of a topological space $X_{\tau}$,\\ $\text{Cl($A$)}=\text{Int($A$)}\cup\text{Bdy($A$)}$. 

Let $A$ be a subset of a topological space $X_{\tau}$ and consider the case where $A = \emptyset$. Then $\text{Cl}(A) = \emptyset$ since $\emptyset$ is both open and closed in $\tau$. And since $\text{Int}(A) \subseteq A \subseteq \text{Cl}(A)$, we have that $\text{Int}(A) = \emptyset$. 
Additionally, since $A = \emptyset$, we have that there are no points $x \in A$ such that a neighborhood $V_x$ of $x$ does not satisfy the conditions for $x \in \text{Bd}(A)$. That is, $\text{Bd}(A) = \emptyset$.

In this case, $\text{Cl}(A) = \text{Int}(A) \cup \text{Bd}(A)$

Now consider the case where $A \neq \emptyset$ and let $x \in \text{Cl}(A)$. By definition, for some neighborhood $V_x$ in $x$, $V_x \cap A \neq \emptyset$. If $V_x \subseteq A$, then $x \in \text{Int}(A)$. If $V_x$ is not a subset of $A$, then $V_x \cap A \neq \emptyset$ and $V_x \cap (X \setminus A) \neq \emptyset$, and so $x \in \text{Bd}(A)$. Thus, 
\[\text{Cl}(A) \subseteq \text{Int}(A) \cup \text{Bd}(A)\]

Now let $x \in \text{Int}(A) \cup \text{Bd}(A)$. If $x \in \text{Int}(A)$, then $x \in \text{Cl}(A)$ since $\text{Int}(A) \subseteq A \subseteq \text{Bd}(A)$.

If $x \in \text{Bd}(A)$, then any neighborhood $V_x$ of $x$ contains points in $A$ and $X \setminus A$. That is, $V_x \cap A \neq \emptyset$ and $V_x \cap (X \setminus A) \neq \emptyset$. That is, $x$ is a limit point of $A$, and thus $x \in \text{Cl}(A)$.

So we have
\[\text{Int}(A) \cup \text{Bd}(A) \subseteq \text{Cl}(A)\]
By double inclusion, we have
\[\text{Cl}(A) = \text{Int}(A) \cup \text{Bd}(A)\]


\item (\#4 in 4.5) Let $A\subseteq X_{\tau}$ and let $f:X_{\tau} \to Y_{\nu}$ be continuous. If $x$ is a limit point of $A$, must $f(x)$ be a limit point of $f(A)\subseteq Y$? Explain.

No. Consider the function $f : \mathbb{R}_{\mathcal{U}} \to \mathbb{R}_{\mathcal{U}}$ defined by 
\[f(x) = 4\]
and let $A = [0,1]$. Notice that $A' = [0,1]$ and that $f(A) = \{4\}$. Also notice that $f(A)' = \emptyset$.

Notice that $\frac{1}{2}$ is a limit point of $A$, and that $f(\frac{1}{2}) = 4 \notin f(A)'$. But $f$ is continuous because it is a constant function. More specifically, if $U \subset \mathbb{R}$ and $\{4\} \in U$, $f^{-1}(U) = \mathbb{R}$ which is open, and if $\{4\} \notin U$, $f^{-1}(U) = \emptyset$ which is open. 

So a limit point of $A$ is not a limit point of $f(A)$. 

\item (\#2 in 5.2) Consider the product space $\mathbb{R}_{\mathcal{L}}\times \mathbb{R}_{\mathcal{L}}$.
\begin{enumerate}
\item Sketch a typical basis set in $\mathbb{R}_{\mathcal{L}}\times \mathbb{R}_{\mathcal{L}}$.
\item Sketch several open sets in $\mathbb{R}_{\mathcal{L}}\times \mathbb{R}_{\mathcal{L}}$.
\item Sketch several sets which are not open in $\mathbb{R}_{\mathcal{L}}\times \mathbb{R}_{\mathcal{L}}$.
\end{enumerate}

See attached sketches.

\item (\#5 in 5.2) Prove Theorem 5.2.3: If $X_{\tau}$ and $Y_{\sigma}$ are any topological spaces, with basepoints $x_0\in X$ and $y_0\in Y$, then the inclusion maps $$i_X:X_{\tau}\hookrightarrow X_{\tau}\times Y_{\sigma}$$ and $$i_Y:Y_{\sigma}\hookrightarrow X_{\tau}\times Y_{\sigma}$$ are both continuous, where $X_{\tau}\times Y_{\sigma}$ denotes the Cartesian product endowed with the product topology.

(Hint: these are maps into a product space).

Let $i_X : X_{\tau} \hookrightarrow X_{\tau} \times Y_{\sigma}$ and $i_Y : Y_{\sigma} \hookrightarrow X_{\tau} \times Y_{\sigma}$ where $X_{\tau} \times Y_{\sigma}$ is the Cartesian product with the product topology. 

Let $U$ be $\tau$-open and consider $(P_X \circ i_X)^{-1}(U)$:
\[(P_X \circ i_X)^{-1}(U)\]
\[ = (i_X^{-1} \circ P_X^{-1})(U)\]
\[ = i_X^{-1}(U \times Y)\]
\[ = U\]
which is open by assumption, so $P_X \circ i_X$ is continuous. Now we wish to show that $P_Y \circ i_X$ is continuous. Let $V$ be $\sigma$-open and consider
\[(P_Y \circ i_X)^{-1}(V)\]
\[ = (i_X^{-1} \circ P_Y^{-1})(V)\]
\[ = i_X^{-1}(X \times V)\]
\[ = X\]
which is open by definition. So $i_X$ is continuous. 

Now we wish to show that $i_Y$ is continuous. Following the same logic above, let $U$ be $\tau$-open and consider
\[(P_X \circ i_Y)^{-1}(U)\]
\[ = (i_Y^{-1} \circ P_X^{-1})(U)\]
\[ = i_Y^{-1}(U \times Y)\]
\[ = Y\]
which is open by definition. Now let $V$ be $\sigma$-open and consider
\[(P_Y \circ i_Y)^{-1}(V)\]
\[ = (i_Y^{-1} \circ P_Y^{-1})(V)\]
\[ = i_Y^{-1}(X \times V)\]
\[ = V\]
which is open by assumption. So $i_Y$ is continuous.




\item Let $f: A \to B$ and $g: C \to D$ be continuous functions. Define a map $f\times g: A\times C \to B\times D$ by the equation $$(f\times g)(a,c) = (f(a), g(c)).$$ Show that $f\times g$ is continuous.

Proof: Let $f: A \to B$ and $g: C \to D$ be continuous functions and let $\beta$ be open in the product topology on $B \times D$. 
Also consider the projection maps
\[P_B : B \times D \to B\]
\[P_D : B \times D \to D\]
To show $f \times g$ is continuous, it suffices to show that $P_B \circ (f \times g)$ and $P_D \circ (f \times g)$ are continuous. 

To begin, we will show that 
\[(f \times g)^{-1} = f^{-1} \times g^{-1}\]
where
\[f^{-1} \times g^{-1}:\]
Let $V_1 \subseteq B$ and $V_2 \subseteq D$.
Since $V_1 \subseteq B$, for some $U_1 \subseteq A$, we have that $f^{-1}(V_1) = U_1$, and similarly for $V_2 \subseteq D$, for some $U_2 \subseteq C$, $g^{-1}(V_2) = U_2$.

Let $\beta = V_1 \times V_2$ and $\alpha = U_1 \times U_2$.

We wish to show that $((f \times g) \circ (f^{-1} \times g^{-1}))(\beta) = \beta$ and $((f^{-1} \times g^{-1}) \circ (f \times g))(\alpha) = \alpha$.

Notice that
\[((f \times g) \circ (f^{-1} \times g^{-1}))(\beta)\]
\[=(f \times g) \circ (f^{-1}(V_1) \times g^{-1}(V_2))\]
\[=(f \times g) (U_1 \times U_2)\]
\[= (f(U_1) \times g(U_2)) = V_1 \times V_2 = \beta\]

Also notice that
\[((f^{-1} \times g^{-1}) \circ (f \times g))(\alpha)\]
\[= (f^{-1} \times g^{-1}) \circ (f(U_1) \times g(U_2))\]
\[= (f^{-1} \times g^{-1}) (V_1 \times V_2)\]
\[= (f^{-1}(V_1) \times g^{-1}(V_2)) = U_1 \times U_2 = \alpha\]

So $(f \times g)^{-1} = (f^{-1} \times g^{-1})$. Now we wish to show that $P_B \circ (f \times g)$ is continuous. Let $U_3 \subseteq B$ and consider
\[(P_B \circ (f \times g))^{-1}(U_3)\]
\[ = ((f \times g)^{-1} \circ P_B^{-1})(U_3)\]
\[ = (f^{-1} \times g^{-1}) (U_3 \times D)\]
\[ = (f^{-1}(U_3) \times g^{-1}(D))\]
\[ = V_3 \times C \subseteq A \times C\]
for some $V_3 \subseteq A$. So $P_B \circ (f \times g)$ is continuous.

Now let $V_4 \subseteq D$ and consider
\[(P_D \circ (f \times g))^{-1}(V_4)\]
\[ = ((f \times g)^{-1} \circ P_D^{-1})(V_4)\]
\[ = (f^{-1} \times g^{-1})(B \times V_4)\]
\[ = (f^{-1}(B) \times g^{-1}(V_4)) = A \times U_4 \subseteq A \times C\]
for some $U_4 \subseteq C$. So $P_D \circ (f \times g)$ is continuous, and so $f \times g$ is continuous.





\end{enumerate}



\noindent \textbf{Bonus}  (\#1 in 5.3) Prove that the basis for the box topology on $\prod X_{\alpha}$, $\mathcal{B}=\prod \tau_{\alpha}$, is in fact a basis. That is, show that it satisfies the two conditions of Definition 4.2.1.
\end{document}
