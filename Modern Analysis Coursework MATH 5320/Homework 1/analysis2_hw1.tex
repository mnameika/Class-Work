\documentclass{article}
\usepackage[utf8]{inputenc}
\usepackage{amssymb}
\usepackage{graphicx}
\graphicspath{{images/}}


\title{Homework 1 (Analysis)}
\author{Michael Nameika}
\date{January 2022}

\begin{document}

\maketitle

\section{}
(a) Let $\{a_n\}$ be a sequence of real numbers such that $|a_{n+1}-a_n| < 3^{-n}$ for all $n \in \mathbb{N}$. Prove that ${a_n}$ is a convergent sequence.
\newline\newline
Proof: We will show that ${a_n}$ satisfies the Cauchy criterion. First, fix $\epsilon > 0$. Now take $m>n>N\in \mathbb{N}$ and consider $|a_m-a_n|$.
\newline\newline
Notice that 
\begin{center}
    $|a_m-a_n| = |a_m-a_{m-1}+a_{m-1}- \ldots + a_{n+1} - a_n|$
\end{center}
And by the triangle inequality,
\begin{center}
    $|a_m-a_{m-1}+a_{m-1}- \ldots + a_{n+1} - a_n| \leq |a_m-a_{m-1}| + |a_{m-1} - a_{m-2}| + \ldots + |a_{n+1} - a_n|$
    \newline\newline
    $< 3^{-(m-1)} + 3^{-(m-2)} + \ldots + 3^{-n}$\
    \newline\newline
    \[=\sum_{k=n}^{m-1}\frac{1}{3^k} \]
    \[=\sum_{k=0}^{m-1}\frac{1}{3^k} - \sum_{k=0}^{n-1}\frac{1}{3^k}\]
    \[=\frac{1-\frac{1}{3^m}}{1-\frac{1}{3}} - \frac{1-\frac{1}{3^n}}{1-\frac{1}{3}}\]
    \[=\frac{3}{2}(1-\frac{1}{3^m}-1+\frac{1}{3^n})\]
    \[=\frac{3}{2}(\frac{1}{3^n} - \frac{1}{3^m})\]
    \[=\frac{1}{2}(\frac{1}{3^{n-1}} - \frac{1}{3^{m-1}})\]
    
\end{center}
And since $m > n$, 

\[3^{-(n-1)} > 3^{-(m-1)}\]
Then
\[\frac{1}{3^{n-1}} - \frac{1}{3^{m-1}} > 0  \]

Additionally, since $m>n>N$, 
\[3^{-(N-1)}>3^{-(m-1)}>3^{-(n-1)}>0 \]
So
\[0<\frac{1}{2}(\frac{1}{3^{n-1}} - \frac{1}{3^{m-1}}) < \frac{1}{2}(\frac{1}{3^{n-1}})<\frac{1}{2}(\frac{1}{3^{N-1}})\]
Now let $\epsilon = \frac{1}{2}(\frac{1}{3^{N-1}})$
\newline
Now we have 
\[|a_m-a_n|<\epsilon\]
Satisfying the Cauchy criterion.
\newline
$\therefore {a_n}$ is a convergent sequence.
\begin{flushright}
    $\square$
\end{flushright}

(b) Let $\{a_n\}$ and $\{b_n\}$ be real sequences such that $|a_n-b_n| \leq \frac{1}{n}$ for all $n \in \mathbb{N}$, and $a_n \rightarrow L$. Then prove that $b_n \rightarrow L$.
\newline\newline
Proof: First note that by definition of convergence, for $\epsilon > 0$, and $n>N\in \mathbb{N}$,
\[|a_n-L| < \epsilon\]
We wish to show that for $n>N$,
\[|b_n-L| < \epsilon ^*\]
for some $\epsilon ^* > 0$
\newline\newline
Begin by noticing that 
\[|b_n - L| = |b_n - a_n + a_n - L|\]
And by the triangle inequality,
\[|b_n-a_n+a_n-L| \leq |b_n-a_n| + |a_n-L|\]
\[=|a_n-b_n| + |a_n-L|\]
And assume that $n>N$, then
\[|a_n-b_n| + |a_n-L| < |a_n-b_n| + \epsilon \]
\[\leq \frac{1}{n} + \epsilon\]
And since $n>N$, $\frac{1}{n} < \frac{1}{N}$, so
\[\frac{1}{n}+\epsilon < \frac{1}{N} + \epsilon \]
Let $\epsilon ^* = \frac{1}{N} + \epsilon > 0$
\newline\newline
Finally, we have
\[|b_n-L|< \epsilon ^*\]
and since $\epsilon ^*$ can be made arbitrarily small, by definition of convergence, $b_n \rightarrow L$.
\begin{flushright}
    $\square$
\end{flushright}

\section{}
2. (a) A sequence of real numbers $\{a_n\}$ is defined by $a_1 = 0$ and $a_{n+1} = \sqrt{3a_n+4}, n \geq 1$. Prove that ${a_n}$ is a convergent sequence and find $\lim_{n \to \infty} a_n$. (Hint: Show that $a_n \leq 4 $ for all $n \geq 1$).
\newline\newline
Proof: First I will show $a_n \leq 4$ for all $n \geq 1$ by induction. The base case is obvious ($a_1 = 0 \leq 4)$. Assume this relationship to be true up to some natural number $k$. We must show the relation also holds for $k+1$.
\newline
By the induction assumption,
\[a_k \leq 4\]
\[3a_k \leq 12\]
\[3a_k+4 \leq 16\]
Then by the definition of $a_{n+1}$,
\[a_{k+1}^2 \leq 16\]
\[|a_{k+1}| \leq 4\]
Thus $a_n \leq 4$ for all $n \geq 1$.
\newline
Now I will show ${a_n}$ is a decreasing sequence by induction.
\newline
Base case:
\[a_2 = \sqrt{3*0+4} = \sqrt{4} = 2 \geq 0 = a_1\]
Now assume this relationship to be true up to some natural number $k$. We must show the relationship also holds for $k+1$. By the induction assumption, 
\[a_k \geq a_{k-1}\]
\[3a_k \geq 3a_{k-1}\]
\[3a_k + 4 \geq 3a_{k-1} + 4\]
\[\sqrt{3a_k+4} \geq \sqrt{3a_{k-1}+4}\]
\[a_{k+1} \geq a_k\]
Which tells us that ${a_n}$ is an increasing sequence, provided $a_k \geq -\frac{4}{3}$ for all $k$. 
\newline
Notice that $a_1 = 0$, and ${a_n}$ is increasing for all non-negative terms, so we have that ${a_n}$ is an increasing sequence. Now Since ${a_n}$ is increasing, bounded above by $4$, and clearly bounded below by $0$, by the Monotone Convergence Theorem, ${a_n}$ is a convergent sequence. 
\begin{flushright}
    $\square$
\end{flushright}
Now that we have established that ${a_n}$ is a convergent sequence, let $\lim_{n \to \infty} a_n = a$. We must find $a$. Begin by applying the limit to the recursion relation:
\[\lim_{n \to \infty}a_{n+1} = \lim_{n \to \infty} \sqrt{3a_n+4}\]
\[a = \sqrt{3a+4}\]
\[a^2 = 3a+4\]
\[a^2-3a-4 = 0\]
\[(a-4)(a+1) = 0\]
So either $a = 4$ or $a = -1$. But since $a_n \geq 0$ for all $n$, we have that a=4. Thus, 
\[\lim_{n \to \infty} a_n = 4\]
(b) Define a sequence $\{x_n\}$ by $x_{n+1} = 1 - \sqrt{1-x_n}$, $n =$ 0,1,2, ... where $0 < x_0 < 1$. Find $x_2$ and $x_3$ in terms of $x_0$ and prove that the sequence $\{x_n\}$ converges.
\newline\newline
\[x_1 = 1 - \sqrt{1 - x_0}\]
\[x_2 = 1 - \sqrt{1 - x_1} = 1 - \sqrt{1-(1-\sqrt{1-x_0})}\]
\[= 1 - (1-x_0)^{\frac{1}{4}}\]
\[x_3 = 1 - \sqrt{1-x_2} = 1 - \sqrt{1 - (1 - (1-x_0)^{\frac{1}{4}})}\]
\[= 1 - (1-x_0)^{\frac{1}{8}}\]
Proof: I will begin by showing $\{x_n\}$ is bounded. A simple induction argument will show 
\[x_n = 1 - (1-x_0)^{\frac{1}{2^n}}\]
Now consider $f_n(x_0) = 1 - (1 - x_0)^{\frac{1}{2^n}}$ on $0 < x_0 < 1$ and find its extreme values:
\[\frac{df_n}{dx_0} = \frac{-1}{2^n}(1-x_0)^{\frac{1}{2^n}-1}(-1)\]
\[= \frac{1}{2^n}(1-x_0)^{\frac{1}{2^n}-1}\]
Notice that $\frac{df_n}{dx_0}$ contains no zeros on (0,1), so by the Extreme Value Theorem, we know that the extreme values must be at $x_0 = 0$ and $x_0 = 1$. Plugging these values into $f_n$:
\[f_n(0) = 1 - \sqrt{1 - 0} = 1 - 1 = 0\]
\[f_n(1) = 1 - \sqrt{1 - 1} = 1\]
Now we have that $\sup f_n(x_0) = 1$ and $\inf f_n(x_0) = 0$, or in other words, $f_n(x_0)$ is bounded. Then $\{x_n\}$ is bounded. And notice that $\frac{df_n}{dx_0} \geq 0$, so $f_n(x_0)$ is increasing, then $\{x_n\}$ is increasing. Now by the Monotone Convergence Theorem, $\{x_n\}$ converges.
\begin{flushright}
    $\square$
\end{flushright}

\section{}
3. (a) Let $\{a_k\}$ be a real sequence. Define $\sigma_n := \frac{a_1 + a_2 + \ldots + a_n}{n}$. If $\lim a_k = a$, prove that $\lim \sigma_n = a$. Show that the converse is false.
\newline\newline
Proof:We have $\lim{a_k} = a$, so by the definition of limits, for $\epsilon > 0$, $\exists N \in \mathbb{N}$ such that when $k > N$, 
\[|a_k - a |< \epsilon\]
We wish to show for some $\hat{\epsilon} > 0$, 
\[|\sigma_n - a| < \hat{\epsilon} \: whenever \: n > N\]
Well,
\[|\sigma_n - a| = |\frac{a_1 + a_2 + \ldots + a_n}{n} - a|\]
\[= |\frac{a_1 + a_2 + \ldots + a_n - na}{n}|\]
\[= \frac{1}{n} |(a_1 - a) + (a_2 - a) + \ldots + (a_n - a)|\]
\[\leq \frac{1}{n} (|a_1 - a| + |a_2 - a| + \ldots + |a_n - a|)\]
\[< \frac{1}{n} (|a_1 - a| + |a_2 - a| + \ldots + |a_N - a| + (n-N)\epsilon)\]
\[= \frac{1}{n} (|a_1 - a| + |a_2 - a| + \ldots + |a_N - a|)  + \frac{n-N}{n}\epsilon\]
\[\leq \frac{1}{n}(|a_1 - a| + |a_2 - a| + \ldots + |a_N - a|) + \epsilon\]
Now let $A = \max{\{|a_1 - a|, |a_2 - a|, \ldots, |a_N-a|\}}$. Now we have
\[\frac{1}{n}(|a_1 - a| + |a_2 - a| + \ldots + |a_N - a|) + \epsilon \leq \frac{1}{n}(NA) + \epsilon\]
\[= \frac{N}{n}A + \epsilon\]
And by the Archimedean property, for $n$ large, $\frac{1}{n} < \epsilon^*$ for some $\epsilon^* > 0$. Now we have
\[\frac{N}{n}A + \epsilon < NA\epsilon^* + \epsilon\]
Now let $\hat{\epsilon} = NA\epsilon^* + \epsilon$, which can be made arbitrarily small. We finally have
\[|\sigma_n - a| < \hat{\epsilon}\]
And by the definition of the limit, $\lim{\sigma_n} = a$.
\begin{flushright}
    $\square$
\end{flushright}
Now consider $\{a_n\} = (-1)^n$. Then $\sigma_n = \frac{-1 + 1 -1 + 1 - \ldots + 1}{n}$. Notice that 
\[\sigma_n = 0 \:n \:even\]
\[\sigma_n = \frac{-1}{n} \:n \:odd\]
Clearly,
\[\lim{\sigma_n} = 0\]
but $\lim{a_n}$ DNE. Thus, the converse is false.
\begin{flushright}
    $\square$
\end{flushright}
(b) For a real sequence $\{a_n\}$ define $d_n := a_{n+1} - a_n $ for $n \geq 1$. If $\lim nd_n = 0$ and the sequence $\{\sigma_n\}$ defined in part (a) converges, then prove that the sequence $\{a_n\}$ converges and $\lim a_n = \lim \sigma_n$. (Hint: Show that $\frac{1}{n}\sum_{k=1}^{n-1}kd_k = a_n - \sigma_n$ for $n > 1$).
\newline\newline
Proof: I will begin by showing that 
\begin{equation}
    \frac{1}{n}\sum_{k=1}^{n-1}kd_k = a_n - \sigma_n
\end{equation}

Using the definition of $\sigma_n$ in part a),
\[a_n - \sigma_n = a_n - \frac{a_1 + a_2 + \ldots + a_n}{n}\]
\[ = \frac{-a_1 - a_2 - \ldots + (n-1)a_n}{n}\]
Now let's expand the left side of equation (1):
\[\frac{1}{n}\sum_{k=1}^{n-1}kd_k = \frac{1}{n}(d_1 + 2d_2 + \ldots + (n-1)d_{n-1})\]
\[ = \frac{1}{n}(a_2 - a_1 + 2a_3 - 2a_2 + \ldots + (n-1)a_{n})\]
\[ = \frac{-a_1 - a_2 - \ldots + (n-1)a_{n}}{n}\]
\[ = a_n - \sigma_n\]

Now, since we are given that $\lim{nd_n} = 0$, from part a), we know that 
\[\lim{(\frac{d_1+2d_2+\ldots+nd_n}{n})} = 0\]
Now let's add and subtract $nd_n$ to the left side of (1):
\[\frac{1}{n}\sum_{k=1}^{n-1}kd_k + nd_n - nd_n = a_n - \sigma_n\]
Which simplifies to 
\[\frac{1}{n}\sum_{k=1}^{n}kd_k - nd_n = a_n - \sigma_n\]
Now apply the limit to each side:
\[\lim{(\frac{1}{n}\sum_{k=1}^{n}kd_k)} - \lim{nd_n}= \lim{(a_n - \sigma_n)}\]
Notice that 
\[\frac{1}{n}\sum_{k=1}^{n}kd_k = \frac{d_1 + 2d_2 + 3d_3 + \ldots + nd_n}{n}\]
And since 
\[\lim{\frac{d_1 + 2d_2 + \ldots + nd_n}{n}} = 0,\]
\[\lim{\frac{1}{n}\sum_{k=1}^{n} kd_k} = 0 \]
Now we have
\[-\lim{nd_n} = \lim{(a_n - \sigma_n)}\]
And since $\lim{nd_n} = 0$,
\[\lim{(a_n - \sigma_n)} = 0\]
It is not entirely clear that $\{a_n\}$ converges. Assume by contradiction that $\{a_n\}$ diverges. And since $\{\sigma_n\}$ converges, say to $\sigma$, for $\epsilon > 0$,  $\exits n > N \in \mathbb{N}$, such that
\[|\sigma_n - \sigma| < \epsilon\]
Or, alternatively,
\[\sigma - \epsilon < \sigma_n < \sigma + \epsilon\]
And notice that 
\[a_n + \sigma - \epsilon > a_n - \sigma_n > a_n - (\sigma - \epsilon) \]
Now we have 
\[0 = \lim{(a_n - \sigma_n)} > \lim{(a_n - (\sigma +\epsilon))}\]
\[ = \lim{a_n} - \lim{\sigma + \epsilon}\]
\[ = \lim{a_n} - (\sigma - \epsilon)\]
\[ = + \infty\]
So now we get 
\[0 > +\infty\]
A contradiction! Thus, $\{a_n\}$ must converge. In fact,
\[\lim{(a_n - \sigma_n)} = 0\]
\[\lim{a_n} - \lim{\sigma_n} = 0\]
\[\lim{a_n} = \lim{\sigma_n}\]
\begin{flushright}
    $\square$
\end{flushright}

\section{}
4. (a) Let $\{a_n\}$ be a strictly decreasing sequence of positive numbers. Assume $\sum_{n=1}^{\infty}a_n$ converges. Prove that $\lim na_n = 0$. (Hint: Use Cauchy convergence criterion for series).
\newline\newline
Proof: We have that $\sum_{n=1}^{\infty} a_n$ converges, so by Cauchy criterion for series, we have for $\epsilon > 0$, $\exists\: n>m>N \in \mathbb{N}$,
\[|\sum_{k=1}^n a_k - \sum_{k=1}^m a_k| < \epsilon\]
Or, equivalently,
\[|\sum_{k=m+1}^n a_k| < \epsilon\]
We also know that $\{a_n\}$ is a strictly decreasing sequence, so for $k = m+1, m+2. \ldots , n$, $a_n < a_i$ for all $i=k$. Then
\[|\sum_{k=m+1}^n a_k| \geq |\sum_{k=m+1}^n a_n |\]
\[= |a_n(n-m)|\]
\[=|n a_n - ma_n|\]
\[\geq |n a_n| - |ma_n|\]
Additionally, by the test for divergence, we have that $\lim a_n = 0$, or by definition, for some $\epsilon^* > 0$, $n > N^* \in \mathbb{N}$,
\[|a_n - 0| < \epsilon^*\]
From above, we have that
\[|n a_n| - |m a_n| < \epsilon\]
\[|n a_n| < \epsilon + m|a_n|\]
Now take $N^{max} = max\{N, N^*\}$ and assume $n > N^{max}$. Now we have
\[|n a_n| < \epsilon + m \epsilon^*\]
let $\hat{\epsilon} = \epsilon + m \epsilon^*$, which can get arbitrarily small. Now we have
\[|n a_n| < \hat{\epsilon}\]
\[|n a_n - 0| < \hat{\epsilon}\]
And finally, by definition of limits, we have that $\lim n a_n = 0$.
\begin{flushright}
    $\square$
\end{flushright}
(b) Give an example of a strictly decreasing positive sequence $\{b_n\}$ such that $\lim_{n \to \infty} n b_n = 0$, but $\sum_{n=1}^{\infty} b_n \:\: diverges$. You must show the divergence of your example series.
\newline\newline
Consider $b_n = \frac{1}{n\ln{n}}$ for n = 2,3,... and $b_1 = 0$.
Notice that 
\[\lim_{n \to \infty} n b_n = \lim_{n \to \infty} \frac{n}{n\ln{n}}\]
\[= \lim_{n \to \infty} \frac{1}{\ln{n}} = 0\]
I will claim that $\sum_{n=1}^{\infty} b_n$ diverges. Quickly note that $\sum_{n=1}^{\infty} b_n = \sum_{n=2}^{\infty} b_n$ since $b_1 = 0$.
\newline\newline
Proof: Consider $f(x) = \frac{1}{x\ln{(x)}}$ on $x \geq 2$ and note that
\[\int_2^{\infty} f(x) dx \leq \sum_{n=2}^{\infty} \frac{1}{n\ln{n}}\]
Well,
\[\int_2^{\infty}f(x)dx = \lim_{a \to \infty} \int_2^a \frac{1}{x\ln{x}}dx\]
Now, let $u = \ln{x}$, then $du = \frac{1}{x} dx$ and the integral becomes
\[\lim_{a \to \infty} \int_2^a \frac{1}{u} du\]
\[=\lim_{a \to \infty} [\ln{|u|}]|_{\ln{2}}^{\ln{a}}\]
\[=\lim_{a \to \infty}(\ln{(\ln{a})} - \ln{(\ln{2})})\]
\[ = + \infty\]
So by comparison, 
\[\sum_{n=2}^{\infty} \frac{1}{n\ln{n}} \: diverges\] 
\begin{flushright}
    $\square$
\end{flushright}

\section{}
5. (a) Use the Mean Value Theorem to show that $\frac{x}{1+x} \leq \ln{(1+x)} \leq x, \:\: x \geq 0$. Then set $x = \frac{1}{n}$ to obtain
\[\frac{1}{n+1} \leq \ln{(1+\frac{1}{n})} \leq \frac{1}{n}\]
Let $f(y) =\ln{(1+y)}$ and consider the interval $y \in [0, x], x > 0$. By the Mean Value Theorem, there exists a $c \in [0,x]$ such that
\[\frac{f(x) - f(0)}{x - 0} = f'(c)\]
Well, 
\[f'(c) = \frac{1}{1+c}\]
So now we have
\[\frac{\ln{(1+x)}}{x} = \frac{1}{1+c}\]
Notice since $c \geq 0$, $\frac{1}{1+c} \leq 1$. Now we have
\[\frac{\ln{(1+x)}}{x} \leq 1\]
\[\ln{(1+x)} \leq x\]
Also notice that since $c \in [0,x]$, $x \geq c$, and 
\[\frac{1}{1+c} \geq \frac{1}{1+x}\]
Now,
\[\frac{1}{1+x} \leq \frac{\ln{(1+x)}}{x}\]
\[\frac{x}{1+x} \leq \ln{(1+x)}\]
Putting these inequalities together, we have
\[\frac{x}{1+x} \leq \ln{(1+x)} \leq x\]
Now replace $x$ with $\frac{1}{n}$ to obtain
\[\frac{\frac{1}{n}}{1+\frac{1}{n}} \leq \ln{(1+\frac{1}{n})} \leq \frac{1}{n}\]
Which simplifies to 
\[\frac{1}{n+1} \leq \ln{(1+\frac{1}{n})} \leq \frac{1}{n}\]


(b) Define $\gamma_n = (1 + 1/2 + 1/3 + \ldots + 1/n) - \ln{n}$. Use part (a) to show that $\gamma_n \geq 0$ and that $\{\gamma_n\}$ is a decreasing sequence.
\newline\newline
Proof: Consider the sum
\[\sum_{k=1}^{n-1} \ln{(1+\frac{1}{k})}\]
And notice that 
\[\ln{(1+\frac{1}{k})} = \ln{(\frac{k+1}{k})}\]
\[= \ln{(k+1)} - \ln{k}\]
Substituting this into the sum above, we get
\[\sum_{k=1}^{n-1} (\ln{(k+1)} - \ln{k}) = \ln{2} - \ln{1} + \ln{3} - \ln{2} + \ldots + \ln{n} - \ln{(n-1)} \]
\[= \ln{n}\]
And by the inequality in part (a), 
\[\sum_{k=1}^{n-1}(\ln{(k+1)}-\ln{k}) \leq \sum_{k=1}^{n-1} \frac{1}{k}\]
\[\ln{n} \leq 1 + \frac{1}{2} + \ldots + \frac{1}{n-1}\]
\[\ln{n} + \frac{1}{n} \leq 1 + \frac{1}{2} + \ldots + \frac{1}{n-1} + \frac{1}{n}\]
\[0 \leq \frac{1}{n} \leq 1 + \frac{1}{2} + \ldots + \frac{1}{n} - \ln{n}\]
So, from above, we have that 
\[\gamma_n \geq 0\]
\begin{flushright}
    $\square$
\end{flushright}
Now we must show $\{\gamma_n\}$ is a decreasing sequence. That is, we must show $\gamma_{n+1} - \gamma_n \leq 0$. By definition of $\gamma_n$, 
\[\gamma_{n+1} - \gamma_n = (1 + \frac{1}{2} + \ldots + \frac{1}{n} + \frac{1}{n+1}) - \ln{(n+1)} - [(1+\frac{1}{2} + \ldots + \frac{1}{n}) - \ln{n}]\]
\[=\frac{1}{n+1} - \ln{(n+1)} + \ln{n}\]
\[=\frac{1}{n+1} - \ln{(1+\frac{1}{n})}\]
Notice by the inequality in part (a),
\[\frac{1}{n+1} \leq \ln{(1+\frac{1}{n})}\]
\[\frac{1}{n+1} - \ln{(1+\frac{1}{n})} \leq 0\]
Then $\gamma_{n+1} - \gamma_n \leq 0$, meaning that $\{\gamma_n\}$ is a decreasing sequence.
\begin{flushright}
    $\square$
\end{flushright}


(c) Show that $\{\gamma_n\}$ converges. $\lim \gamma_n = \gamma$ is called Euler's constant.
\newline\newline
Since $\{\gamma_n\}$ is a decreasing sequence, $\gamma_1$ will be an upper bound for $\{\gamma_n\}$.
\[\gamma_1 = 1 - \ln{1} = 1\]
And since we showed that $\gamma_n \geq 0$ for all $n$, 
\[0 \leq \gamma_n \leq 1\]
Now we have that $\{\gamma_n\}$ is a bounded decreasing sequence, and so, by the Monotone Convergence Theorem, $\{\gamma_n\}$ converges.
\begin{flushright}
    $\square$
\end{flushright}
\end{document}