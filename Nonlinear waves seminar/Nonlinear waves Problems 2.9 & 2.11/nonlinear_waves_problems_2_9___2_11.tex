\documentclass{article}
\usepackage{graphicx, amsmath, mathtools, amssymb, fancyhdr} % Required for inserting images

\setlength{\oddsidemargin}{0in}
\setlength{\textwidth}{6.5in}
\setlength{\topmargin}{-.55in}
\setlength{\textheight}{9in}
\pagestyle{fancy}





\begin{document}

\begin{center}
    {\huge Nonlinear Waves Problems 2.9 \& 2.11}
    \vspace{0.5cm}

    {\large Michael Nameika}
\end{center}

\begin{itemize}
    \item[2.9] Consider the equation
    \[u_t + uu_x = 1, \hspace{0.3cm} -\infty < x < \infty, \hspace{0.3cm} t > 0. \]
    \begin{itemize}
        \item[(a)] Find the general solution.
        \newline\newline
        \textit{Soln.} By method of characteristics, we wish to solve the following system of ODEs:
        \begin{align*}
            \frac{dx}{ds} &= u\\
            \frac{dt}{ds} &= 1\\
            \frac{du}{ds} &= 1.
        \end{align*}
        From these equations, we can immediately see $u(s) = s + c_1$ and $t(s) = s + c_2$. From these two equations, we have $s = t - c_2$ and so $u = t-c_2 + c_1 = \tilde{c}$ with $\tilde{c} = c_1 - c_2$. That is, $u - t = \text{constant}$. And from $\frac{dx}{ds}$, we find $x(s) = \frac{t^2}{2} + \tilde{c}t + c_3$, where $c_1,c_2,c_3 \in \mathbb{R}$. Further, adding and subtracting $\frac{s^2}{2}$ from $x(s)$ gives us $x(s) = -\frac{s^2}{2} + su + c_3 \implies x +\frac{s^2}{2}-su = c_3$. Relating the constants $\tilde{c}$ and $c_3$ with some arbitrary function $g$, we have $g(c_3) = \tilde{c} \implies u-t = g\left(x + \frac{t^2}{2}-tu \right).$ Assuming initial condition $u(x,0) = f(x)$, it is easy to see 
        \[u = t + f\left(x + \frac{t^2}{2}-tu\right).\]


        \item[(b)] Discuss the solution corresponding to: $u = \frac{1}{2}t$ when $t^2 = 4x$.
        \newline\newline
        \textit{Soln.} Not a solution.



        \item[(c)] Discuss the solution corresponding to: $u = t$ when $t^2 = 2x$.
        \newline\newline
        \textit{Soln.} From our work in determining the characteristics in part (a), we can see that the case $u = t$ corresponds to $\tilde{c} = 0$ and $t^2 = 2x$ implies $c_3 = 0$. This also corresponds to the case $f(x) = 0$ and so we can conclude $u = t$ on the curve $t^2 = 2x$.
    \end{itemize}


    \item[2.11] Find the solution of the equation
    \[yu_x - xu_y = 0,\]
    corresponding to the data $u(x,0) = f(x)$. Explain what happens if we give $u(x(s),y(s)) = f(s)$ along the curve defined by $\{s : x^2(s) + y^2(s) = a^2\}.$
    \newline\newline
    \textit{Soln.} By method of characteristics, we wish to solve the following set of differential equations:
    \begin{align*}
        \frac{dx}{ds} &= y\\
        \frac{dy}{ds} &= -x\\
        \frac{du}{ds} &= 0.
    \end{align*}
    From this system, we conclude $u = c_1 \in \mathbb{R}$, $x = A_1\cos(s) + A_2\sin(s)$, and $y = B_1\cos(s) + B_2\sin(s)$. From the initial condition $u(x,0) = f(x)$, we take $x_0 = A_1$, $y_0 = 0$. Solving for the constants $A_1,A_2,B_1,B_2$ yields $A_1 = t$, $A_2 = 0$, $B_1 = 0$, $B_2 = t$. Giving $x = A_1\cos(s)$ and $y = A_1\sin(s)$. Thus $x^2 + y^2 = A_1^2 = \text{constant}$. Therefore the characteristics are circles centered at the origin, and $u = \text{constant}$ on these characteristics. Further, we can relate the constants $t^2$ and $c_1$ by an arbitrary function $g(A_1^2) = c_1$, so that 
    \[u(x,y) = g(x^2 + y^2).\]
    By our initial condition $u(x,0) = f(x)$, we have
    \begin{align*}
        g(x^2) &= f(x)\\
        \implies f\left(\pm\sqrt{x}\right) &= g(x)\\
        \implies f\left(\pm\sqrt{x^2 + y^2}\right) &= g(x^2 + y^2).
    \end{align*}
    Thus the general solution to this PDE is given as
    \[u(x,y) = f\left(\sqrt{x^2 + y^2}\right).\]
    Now, on the characteristic $\{s : x^2(s) + y^2(s) = a^2\}$, we have $u(x(s),y(s)) = f(s) = f(a)$.
\end{itemize}

\end{document}
