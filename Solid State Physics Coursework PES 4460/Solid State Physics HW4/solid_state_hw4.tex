\documentclass{article}
\usepackage[utf8]{inputenc}
\usepackage{amsmath}
\usepackage{amssymb}

\setlength{\oddsidemargin}{0in}
\setlength{\textwidth}{6.5in}
\setlength{\topmargin}{-.55in}
\setlength{\textheight}{9in}
\pagestyle{empty}



\title{Solid State Physics HW4}
\author{Michael Nameika}
\date{October 2022}

\begin{document}

\maketitle

\section*{Problem 1}

From Table 1 in Chapter 3 of Kittel, we see that copper and gold have similar cohesive energies, and likewise from Table 2 in Chapter 1, we see that copper and gold have similar lattice constants and both crystallize in the face centered cubic system. That is, it is reasonable to expect that a copper-gold alloy will crystallize in the face centered cubic system as well. By the attached drawings, we can see that the expected concentrations of copper-gold are 3:1, 1:1, 1:3. That is, we expect alloys of concentration of $25\%/75\%$, $50\%/50\%$, and $75\%/25\%$ copper/gold.

For an experiment to determine if the alloy is ordered or not, 


\section*{Problem 2}

\textbf{\textit{Structure factor of diamond.}}
\newline
(a) Find the structure factor $S$ of this basis.
\newline
Recall the formula for the structural factor
\[S = \sum_i f_i e^{-i\textbf{r}_i\cdot \textbf{G}}\]
where we sum over all atoms of the basis. Recall that the crystal structure of diamond is face centered cubic with an additional face centered cubic lattice displaced by $\left(\frac{1}{4},\frac{1}{4}, \frac{1}{4}\right)$ from the original. Further recall that a face centered cubic lattice has 4 basis atoms, so the diamond crystal structure will have 8 basis atoms. Additionally, since diamond consists only of carbon atoms, each $f_i$ is the same, so we may factor it out of the sum. Then our structure factor for diamond takes the following form:
\[S_{\text{diamond}} = f(1 + e^{-i\pi(v_1 + v_2)} + e^{-i\pi(v_1 + v_3)} + e^{-1\pi(v_2 + v_3)} + e^{-i\pi/2(v_1 + v_2 + v_3)} + e^{-i\pi/2(3v_1 + 3v_2 + v_3)} + \cdots  \]
\[e^{-i\pi/2(3v_1 + v_2 + 3v_3)} + e^{-i\pi/2(v_1 + 3v_2 + 3v_3)})\]
(b) Find the zeros of $S$ and show that the allowed reflections of the diamond structure satisfy $v_1 + v_2 + v_3 = 4n$, where all indices are even and $n$ is any integer, or else all indices are odd.
\newline
Notice we may rewrite the equation for $S_{\text{diamond}}$ in the following way:
\[S_{\text{diamond}} = f (1 + e^{-i\pi(v_1 + v_2)} + e^{-i\pi(v_1 + v_3)} + e^{-1\pi(v_2 + v_3)} + e^{-i\pi/2(v_1 + v_2 + v_3)}(1 + e^{-i\pi(v_1 + v_2)} + e^{-i\pi(v_1 + v_3)} + e^{-1\pi(v_2 + v_3)}))\]
factoring, we find
\[S_{\text{diamond}} = f(1 + e^{-i\pi(v_1 + v_2)} + e^{-i\pi(v_1 + v_3)} + e^{-1\pi(v_2 + v_3)})(1 + e^{-i\pi/2(v_1 + v_2 + v_3)})\]
Notice that the first factor is simply the structure factor for a face centered cubic lattice, and so is nonzero if and only if all $v_1,v_2,v_3$ are either odd or even. Now we must find when the second factor, $1 + e^{-i\pi/2(v_1 + v_2 + v_3)}$ is equal to zero given the above condition.
\newline

\textbf{Case 1}: $v_1 = 2h$, $v_2 = 2k$, $v_3 = 2l$ for some $h,k,l \in \mathbb{Z}$.
\newline
Then 
\[1 + e^{-i\pi/2(v_1 + v_2 + v_3)} = 1 + e^{-i\pi(h + k + l)}\]
Notice that if $h + k + l = 2m + 1$ for some $m \in \mathbb{Z}$, then the above equation equals zero. Likewise, if $h + k + l = 2n$ for some $n \in \mathbb{Z}$, then the above equation is nonzero. Then the structure factor is nonzero whenever
\[v_1 + v_2 + v_3 = 2h + 2k + 2l = 2(h + k + l) = 2(2n) = 4n\]
for some $n \in \mathbb{Z}$.
\newline

\textbf{Case 2}: $v_1 = 2h + 1$, $v_2 = 2k + 1$, $v_3 = 2l + 1$, for some $h,k,l \in \mathbb{Z}$.
\newline
Then $v_1 + v_2 + v_3 = 2(h + k + l + 1) + 1$ and
\[1 + e^{-i\pi/2(v_1 + v_2 + v_3)} = 1 + e^{-i\pi/2(2(h + k + l + 1) + 1)}\]
\[ = 1 + e^{-i\pi(h + k + l + 1)}e^{-i\pi/2}\]
\[ = 1 \pm i\]


So we can see that the structure factor for diamond is nonzero whenever all indices are odd, or $v_1 + v_2 + v_3 = 4n$ for some $n \in \mathbb{Z}$ whenever $v_1,v_2,v_3 \in 2\mathbb{Z}$.

\section*{Problem 3}

Consider ceramic material of AB type with a known mass density of 2.65 $\text{g/cm}^3$ and a unit cell of cubic symmetry with a lattice constant of 0.43 nm. The atomic weights of the A and B elements are 86.6 and 40.3 g/mol, respectively. Determine the possible crystal structures for this material among the rock salt, cesium chloride, and zincblende structures.
\newline

We are given the material has a mass density $\rho = 2.65 \text{g/cm}^3$ with cubic symmetry with a lattice constant of $a = 4.3 \text{\AA}$ where the atomic weight of the A element is $86.6 \text{g/mol}$ and the atomic weight of the B element is $40.3 \text{g/mol}$. To begin, let us convert the mass density from $\text{g/cm}^3$ to $\text{g/\AA}^3$. Recall an angstrom is $10^{-10} \text{m}$ and a centimeter is $10^{-2} \text{m}$. Then it follows that there are $10^8 \text{\AA}$ per centimeter. Then there are $10^{24} \text{\AA}^3$ per $\text{cm}^3$. Now our mass density becomes
\begin{align*}
    \rho &= 2.65 \frac{\text{g/cm}^3}{10^{24} \text{\AA}^3/\text{cm}^3} \\
    &= 2.65 \times 10^{-24} \text{g/\AA}^3
\end{align*}
Now, let us find the atomic mass of A and B in terms of g/atoms. Recall that $1 \text{mol} = 6.02 \times 10^{23} \text{atoms}$. Denote the atomic weight of A and B as $w_A$ and $w_B$, respectively. Then
\begin{align*}
    w_A &= \frac{86.6 \text{g}}{6.02\times 10^{23} \text{atoms}} \\
    &= 1.43854\times 10^{-22} \text{g/atom} \\
    w_B &= \frac{40.3 \text{g}}{6.02 \times 10^{23} \text{atoms}} \\
    &= 6.69435 \times 10^{-23} \text{g/atom} \\
\end{align*}
Now, since the material is of cubic symmetry, the volume of a unit cell is simply given by $V = a^3$. Then the volume of the unit cell is
\begin{align*}
    V &= (4.3 \text{\AA})^3 \\
    &= 79.507 \text{\AA}^3 \\
\end{align*}
And our mass per unit cell is given by
\begin{align*}
    m &= \rho V \\
    &= (2.65 \times 10^{-24} \text{g/\AA}^3)(79.507 \text{\AA}^3) \\
    &= 2.107 \times 10^{-22} \text{g} \\
\end{align*}
Now that we have the atomic mass of A and B, and the mass per unit cell, we may analyze the rock salt, zincblende, and cesium chloride structures to determine which structure this material crystallizes in. 
Let us begin with the rock salt structure. Recall that the rock salt structure is face centered cubic with an additional face centered cubic lattice displaced by $(1/2,1/2,1/2)$, and thus will have 4 A and 4 B atoms in the basis. 
That is, the total mass of the rock salt structure for the AB material will be given by
\begin{align*}
    m_{rs} &= 4w_A + 4w_B \\
    &= 4(1.43854 \times 10^{-22} \text{g}) + 4(6.69435 \times 10^{-23} \text{g}) \\
    &= 5.75416 \times 10^{-22} \text{g} + 2.67774 \times 10^{-22} \text{g} \\
    &= 8.4319 \times 10^{-22} \text{g} \\
    &\neq m \\
\end{align*}
So AB does not crystallize in the rock salt structure. Now let us analyze the zincblende structure. 
Recall that zincblende crystallizes in the diamond crystal structure, which is a face centered cubic lattice with an additional face centered cubic lattice displaced by $(1/4,1/4,1/4)$. 

Then there are 4 A atoms and 4 B atoms per basis, the same number of basis atoms as in the rock salt structure. 
Then the mass of AB in the zincblende structure will the same as in the rock salt structure, so AB does not crystallize in the zincblende structure.

Finally, let us analyze the cesium chloride structure. Recall that the cesium chloride structure is a simple cubic lattice with an additional simple cubic lattice displaced by $(1/2,1/2,1/2)$.
Then there will be 1 A and 1 B atom per unit cell. Denote the mass of AB in the cesium chloride structure as $m_{CsCl}$ and let us calculate the mass of this unit cell:
\begin{align*}
    m_{CsCl} &= w_A + w_B \\
    &= 1.43854 \times 10^{-22} \text{g} + 6.69435 \times 10^{-23} \text{g} \\
    &= 2.10798 \times 10^{-22} \text{g} \\
    &\approx m \\
\end{align*}
So we can see that the AB material crystallizes in the cesium chloride structure.
\newline


\section*{Problem 4}
At 300K, Al has an FCC structure with a lattice constant of 0.405 nm. The coefficient of thermal expansion for Al is $\alpha = 25 \times 10^{-6} \text{K}^{-1}$. During the diffraction experiment, we employed $\text{CuK}_{\alpha 1}$ radiation which has a wavelength of $\lambda = 0.15406$ nm. Our goal was to study the change $\Delta \theta$ in $\theta $ due to the change of temperature from 300 to 600K.
\begin{enumerate}
    \item Derive an expression for $\Delta \theta$ in terms of $\theta$. 
    \newline
    From Bragg's law, we have $\sin{(\theta)} = \frac{\lambda\sqrt{h^2 + k^2 + l^2}}{2a}$. Differentiating with respect to $T$ and using the fact that $\frac{da}{dT} = \alpha a$ (thermal expansion), we find
    \begin{align*}
        \cos{(\theta)}\frac{d\theta}{dT} &= \frac{\lambda\sqrt{h^2 + k^2 + l^2}}{2}(-a^{-2})\frac{da}{dT} \\
        \frac{d\theta}{dT} &= -\sec{(\theta)}\frac{\lambda\sqrt{h^2+k^2+l^2}}{2a}(\alpha) \\
        \Delta \theta &= -\sec{(\theta)}\frac{\lambda \sqrt{h^2 + k^2 + l^2}}{2a}(\alpha \Delta T) \\
    \end{align*}
    
    \item For the (111), (333), and (442) reflections, evaluate $\Delta \theta$.
    \newline
    To begin, let us calculate the angle that corresponds to the (111) plane. Using Bragg's law, we get
    \begin{align*}
        \theta &= \arcsin{\left(\frac{1.5406(\sqrt{3}) \text{\AA}}{2(4.05) \text{\AA}}\right)} \\
        &\approx 19.23^{\circ} \\
    \end{align*}
    Plugging this value into the equation we found for $\Delta \theta$, we find
    \[\Delta \theta \approx -0.15^{\circ}\]
    That is, the diffraction lines will move approximately $.15^{\circ}$ left. Now, let us find the angle that corresponds to the plane (333):
    \begin{align*}
        \theta &= \arcsin{\left( \frac{1.5406(3\sqrt{3}) \text{\AA}}{8.1 \text{\AA}} \right)} \\
        &\approx 81.23^{\circ} \\
    \end{align*}
    Plugging this value into the equation we found for $\Delta \theta$, we get
    \[\Delta \theta = -0.056^{\circ}\]
    That is, the diffraction lines will move approximately $0.056^{\circ}$ to the left. Finally, let us find the angle that corresponds to the plane (442):
    \newline
    Plugging it in, I found a complex number for the angle. Something is not right here...
    \[\theta = \arcsin{\left(\frac{1.5406*6}{8.1}\right)}\]
    ?
    
    \item Based on the results from the previous two tasks, how would you design the experiment to measure the thermal expansion coefficient for an unknown compound?
    \newline
    
    Suppose we perform an x-ray diffraction experiment on some (cubic) material at some temperature (eg. room temperature), and we calculate the Miller indices. 
    Then suppose we increase the temperature by a fixed amount, and perform the x-ray diffraction experiment again, and we find the difference between the initial temperature's angles and the new temperature's angles. 
    
    Then using the equation we found in part 1, we may rearrange the terms to solve for the thermal expansion coefficient $\alpha$. 
    
\end{enumerate}



\end{document}
