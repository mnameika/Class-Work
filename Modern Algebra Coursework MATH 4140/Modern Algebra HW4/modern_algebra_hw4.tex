\documentclass{article}
\usepackage[utf8]{inputenc}
\usepackage{amsmath}
\usepackage{amssymb}
\usepackage{mathtools}

\setlength{\oddsidemargin}{0in}
\setlength{\textwidth}{6.5in}
\setlength{\topmargin}{-.55in}
\setlength{\textheight}{9in}
\pagestyle{empty}

\title{Modern Algebra HW4}
\author{Michael Nameika}
\date{October 2022}

\begin{document}

\maketitle

\section*{Section 10 Problems}

\textbf{6.} Find all left cosets of the subgroup $\{\rho_0, \mu_2\}$ of the group $D_4$ given by Table 8.12. 
\newline

Let us begin by finding $\rho_0\{\rho_0, \mu_2\}$, $\rho_1\{\rho_0, \mu_2\}$, $\rho_2\{\rho_0, \mu_2\}$, and $\rho_3\{\rho_0, \mu_2\}$:
\begin{align*}
    \rho_0\{\rho_0, \mu_2\} &= \{\rho_0\rho_0, \rho_0\mu_2\} = \{\rho_0, \mu_2\} \\ 
    \rho_1\{\rho_0, \mu_2\} &= \{\rho_1\rho_0, \rho_1\mu_2\} = \{\rho_1, \delta_2\} \\
    \rho_2\{\rho_0, \mu_2\} &= \{\rho_2\rho_0, \rho_2\mu_2\} = \{\rho_2, \mu_1\} \\
    \rho_3\{\rho_0, \mu_2\} &= \{\rho_3\rho_0, \rho_3\mu_2\} = \{\rho_3, \delta_1\} \\
\end{align*}
Recall that cosets of a subgroup $H$ of $G$ partition $G$, so we may observe that we have found all partitions of $G$ under the cosets of $H$. That is, the left cosets of $\{\rho_0, \mu_2\}$ of $D_4$ are the following:
\[\{\rho_0, \mu_2\} \:\:  \:\: \{\rho_1, \delta_2\} \:\:  \:\: \{\rho_2, \mu_1\} \:\:  \:\: \{\rho_3, \delta_1\}\]
To verify this, let's find 
$\mu_1\{\rho_0, \mu_2\}$, $\mu_2\{\rho_0, \mu_2\}$, $\delta_1\{\rho_0, \mu_2\}$, and $\delta_2\{\rho_0, \mu_2\}$:
\begin{align*}
    \mu_1\{\rho_0, \mu_2\} &= \{\mu_1\rho_0, \mu_1\mu_2\} = \{\mu_1, \rho_2\} \\
    \mu_2\{\rho_0, \mu_2\} &= \{\mu_2\rho_0, \mu_2\mu_2\} = \{\mu_2, \rho_0\} \\
    \delta_1\{\rho_0, \mu_2\} &= \{\delta_1\rho_0, \delta_1\mu_2\} = \{\delta_1, \rho_3\} \\
    \delta_2\{\rho_0, \mu_2\} &= \{\delta_2\rho_0, \delta_2\mu_2\} = \{\delta_2, \rho_1\} \\
\end{align*}
Which we notice to be the same as the cosets created by $\rho_0$, $\rho_1$, $\rho_2$, and $\rho_3$.
\newline\newline
\textbf{7.}  Repeat the preceding exercise, but find the right cosets this time. Are they the same as the left cosets?
\newline

As in problem 6, let us begin by finding the right cosets of $\{\rho_0, \mu_2\}$ for each $\rho$:
\begin{align*}
    \{\rho_0, \mu_2\}\rho_0 &= \{\rho_0\rho_0, \mu_2\rho_0\} = \{\rho_0, \mu_2\} \\
    \{\rho_0, \mu_2\}\rho_1 &= \{\rho_0\rho_1, \mu_2\rho_1\} = \{\rho_1, \delta_1\} \\
    \{\rho_0, \mu_2\}\rho_2 &= \{\rho_0\rho_2, \mu_2\rho_2\} = \{\rho_2, \mu_1\} \\
    \{\rho_0, \mu_2\}\rho_3 &= \{\rho_0\rho_3, \mu_2\rho_3\} = \{\rho_3, \delta_2\} \\
\end{align*}
So the right cosets of $\{\rho_0, \mu_2\}$ in $D_4$ are 
\[\{\rho_0, \mu_2\} \:\:\:\: \{\rho_1, \delta_1\} \:\:\:\: \{\rho_2, \mu_1\} \:\:\:\: \{\rho_3, \delta_2\}\]
These are NOT the same cosets as in problem 6. For example, the right coset $\{\rho_1, \delta_1\}$ is not a left coset of $\{\rho_0, \mu_2\}$ in $D_4$.
\newline\newline
\textbf{9.} Repeat Exercise 6 for the subgroup $\{\rho_0, \rho_2\}$ of $D_4$. 
\newline

Let us begin by finding $\mu_1\{\rho_0, \rho_2\}$, $\rho_1\{\rho_0, \rho_2\}$, $\delta_1\{\rho_0, \rho_2\}$, and $\rho_0\{\rho_0, \rho_2\}$:
\begin{align*}
    \mu_1\{\rho_0, \rho_2\} &= \{\mu_1\rho_0, \mu_1\rho_2\} = \{\mu_1, \mu_2\} \\
    \rho_1\{\rho_0, \rho_2\} &= \{\rho_1\rho_0, \rho_1\rho_2\} = \{\rho_1, \rho_3\} \\
    \delta_1\{\rho_0, \rho_2\} &= \{\delta_1\rho_0, \delta_1\rho_2\} = \{\delta_1, \delta_2\} \\
    \rho_0\{\rho_0, \rho_2\} &= \{\rho_0\rho_0, \rho_0\rho_2\} = \{\rho_0, \rho_2\} \\
\end{align*}
So the left cosets of $\{\rho_0, \rho_2\}$ in $D_4$ are 
\[\{\rho_0, \rho_2\} \:\:\:\: \{\mu_1, \mu_2\} \:\:\:\: \{\rho_1, \rho_3\} \:\:\:\: \{\delta_1, \delta_2\}\]
\newline\newline
\textbf{10.} Repeat the preceding exercise, but find the right cosets this time. Are they the same as the left coests?
\newline

Repeating problem 9 to find the right cosets, we find
\begin{align*}
    \{\rho_0, \rho_2\}\mu_1 &= \{\rho_0\mu_1, \rho_2\mu_1\} = \{\mu_1, \mu_2\} \\
    \{\rho_0, \rho_2\}\rho_1 &= \{\rho_0\rho_1, \rho_2\rho_1\} = \{\rho_1, \rho_3\} \\
    \{\rho_0, \rho_2\}\delta_1 &= \{\rho_0\delta_1, \rho_2\delta_1\} = \{\delta_1, \delta_2\} \\
    \{\rho_0, \rho_2\}\rho_0 &= \{\rho_0\rho_0, \rho_2\rho_0\} = \{\rho_0, \rho_2\} \\
\end{align*}
So the right cosets of $\{\rho_0, \rho_2\}$ in $D_4$ are
\[\{\rho_0, \rho_2\} \:\:\:\: \{\mu_1, \mu_2\} \:\:\:\: \{\rho_1, \rho_3\} \:\:\:\: \{\delta_1, \delta_2\}\]
which are the same as the left cosets we fond in exercise 9.
\newline\newline
\textbf{28.} Let $H$ be a subgroup of a group $G$ such that $g^{-1}hg \in H$ for all $g \in G$ and all $h \in H$. Show that every left coset of $gH$ is the same as the right coset $Hg$.
\newline

Proof: Let $H$ be a subgroup of a group $G$ such that $g^{-1}hg \in H$ for all $g \in G$. We will proceed by double inclusion. That is, we will show that $gH \subseteq Hg$ and $Hg \subseteq gH$. Begin by considering $gh \in gH$. Notice that we may rewrite $gh$ as $ghg^{-1}g$ and further as $((g^{-1})^{-1}hg^{-1})g$. Then let $a = g^{-1} \in G$. We find
\[(g^{-1})^{-1}hg^{-1} = a^{-1}ha\]
substituting, we have
\[((g^{-1})^{-1}hg^{-1})g = (a^{-1}ha)g\]
and since $a \in G$, $a^{-1}ha \in H$ by hypothesis, and by definition, $(a^{-1}ha)g \in Hg$. That is, $gh \in Hg$, so we have
\[gH \subseteq Hg\]
Now consider $hg \in Hg$. Similar to above, we may rewrite $hg$ as $hg = gg^{-1}hg = g(g^{-1}hg)$. By hypothesis, $g^{-1}hg \in H$, so $g(g^{-1}hg) \in gH$ by definition. Then we have
\[Hg \subseteq gH\]
Finally, by double inclusion, we have $Hg = gH$.
\newline\newline
\textbf{40.} Show that if a group $G$ with identity $e$ has finite order $n$, then $a^n = e$ for all $a \in G$.
\newline

Proof: Let $G$ be a group with identity $e$ and suppose $|G| = n$ and let $a \in G$. Consider $\langle a \rangle$ and suppose $|\langle a \rangle | = m$ for some $m \in \mathbb{Z}$. That is, $m$ is the smallest integer such that $a^m = e$. By Lagrange's theorem, we have have that $m \big| n$ since $\langle a \rangle \leq G$. Then for some $k \in \mathbb{Z}$, we have $n = mk$. Now consider $a^n$:
\[a^n = a^{mk} = (a^{m})^k = e^k = e\]
Which is what we sought to show.

\section*{Section 11 Problems}

\textbf{1.} List the elements of $\mathbb{Z}_2 \times \mathbb{Z}_4$. Find the order of each of the elements. Is this group cyclic?

The elements of $\mathbb{Z}_2 \times \mathbb{Z}_4$ are as follows:
\[(0,0) \:\: (0,1) \:\: (0,2) \:\: (0,3)\]
\[(1,0) \:\: (1,1) \:\: (1,2) \:\: (1,3)\]
This group is not cyclic. In light of theorem 11.5, we may observe that $\text{gcd}(2,4) = 2$, so $\mathbb{Z}_2 \times \mathbb{Z}_4$ is not cyclic. Alternatively, notice that there are eight elements in $\mathbb{Z}_2 \times \mathbb{Z}_4$, and that each element has the following orders:
\newline
\begin{center}
    \begin{tabular}{c|c}
        $|(0,0)| = 1$ & $|(1,0)| = 2$ \\
        $|(0,1)| = 2$ & $|(1,1)| = 4$ \\
        $|(0,2)| = 2$ & $|(1,2)| = 2$ \\
        $|(0,3)| = 4$ & $|(1,3)| = 4$ \\
    \end{tabular}
\end{center}
Since there are eight elements and the highest order of an element of $\mathbb{Z}_2 \times \mathbb{Z}_4$ is four, we cannot generate the group from a single element, so $\mathbb{Z}_2 \times \mathbb{Z}_4$ is not cyclic.
\newline\newline
\textbf{2.} Repeat Exercise 1 for the group $\mathbb{Z}_3 \times \mathbb{Z}_4$
\newline

The elements of $\mathbb{Z}_3 \times \mathbb{Z}_4$ are as follows:
\[(0,0) \:\: (0,1) \:\: (0,2) \:\: (0,3)\]
\[(1,0) \:\: (1,1) \:\: (1,2) \:\: (1,3)\]
\[(2,0) \:\: (2,1) \:\: (2,2) \:\: (2,3)\]
By theorem 11.5, since $\text{gcd}(3,4) = 1$, we have that $\mathbb{Z}_3 \times \mathbb{Z}_4$ is cyclic. Alternatively, notice that the elements have the following orders:
\begin{center}
    \begin{tabular}{c|c|c|c}
        $|(0,0)| = 1$ &$|(0,1)| = 4$ & $|(0,2)| = 2$ & $|(0,3)| = 4$ \\
        $|(1,0)| = 2$ & $|(1,1)| = 12$ & $|(1,2)| = 6$ & $|(1,3)| = 12$ \\
        $|(2,0)| = 3$ & $|(2,1)| = 12$ & $|(2,2)| = 6$ & $|(2,3)| = 12$ \\
    \end{tabular}
\end{center}
Notice four elements have order 12, and since $\mathbb{Z}_3 \times \mathbb{Z}_4$ has 12 elements, we can see that $\mathbb{Z}_3 \times \mathbb{Z}_4$ is cyclic. 
\newline\newline
\textbf{14.} Fill in the blanks.
\begin{enumerate}
    \item[\textbf{a.}] The cyclic subgroup of $\mathbb{Z}_{24}$ generated by 18 has order 4.
    \newline
    To begin, we must find the greatest common divisor of 24 and 18. Well, $\text{gcd}(24,18) = 6$, so the cyclic subgroup of $\mathbb{Z}_{24}$ generated by 18 has order $24/6 = 4$.
    \newline
    \item[\textbf{b.}] $\mathbb{Z}_3 \times \mathbb{Z}_4$ is of order 12.
    See problem 2. 
    \item[\textbf{c.}] The element $(4,2)$ of $\mathbb{Z}_{12} \times \mathbb{Z}_8$ has order 12.
    \newline
    We must find the least common multiple of the greatest common divisors of 4 and 12, and 2 and 8, respectively. Begin with 4 and 12:
    \[\text{gcd}(4,12) = 4\]
    So 4 has order 12/4 = 3 for $\mathbb{Z}_{12}$. Now for 2 and 8:
    \[\text{gcd}(2,8) = 2\]
    So 2 has order 8/2 = 4 for $\mathbb{Z}_8$. Now we must find the least common multiple between 3 and 4:
    \[\text{lcm}(3,4) = \frac{3\cdot 4}{\text{gcd}(3,4)} = \frac{12}{1} = 12\]
    \item[\textbf{d.}] The Klein 4-group is isomorphic to $\mathbb{Z}_2 \times \mathbb{Z}_2$.
\end{enumerate}

\end{document}
