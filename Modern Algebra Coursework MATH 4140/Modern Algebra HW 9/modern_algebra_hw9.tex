\documentclass{article}
\usepackage[utf8]{inputenc}
\usepackage{amsmath}
\usepackage{amssymb}
\usepackage{mathtools}

\setlength{\oddsidemargin}{0in}
\setlength{\textwidth}{6.5in}
\setlength{\topmargin}{-.55in}
\setlength{\textheight}{9in}
\pagestyle{empty}


\title{Modern Algebra HW 9}
\author{Michael Nameika}
\date{November 2022}

\begin{document}

\maketitle

\section*{Section 20 Problems}
3. Find a generator for the multiplicative group $\mathbb{Z}_{17}$
\newline\newline
I claim that $3 \in \mathbb{Z}_{17}$ is a generator for $\langle \mathbb{Z}_{17}^{\neq 0}, \cdot \rangle$. To see this, notice the following:
\begin{align*}
    3 \times 3 &= 9 \mod 17 \\
    9 \times 3 &= 10 \mod 17 \\
    10 \times 3 &= 13 \mod 17 \\
    13 \times 3 &= 5 \mod 17\\
    5 \times 3 &= 15 \mod 17\\
    15 \times 3 &= 11 \mod 17\\
    11 \times 3 &= 16 \mod 17\\
    16 \times 3 &= 14 \mod 17\\
    14 \times 3 &= 8 \mod 17\\
    8 \times 3 &= 7 \mod 17\\
    7 \times 3 &= 4 \mod 17\\
    4 \times 3 &= 12 \mod 17\\
    12 \times 3 &= 2 \mod 17\\
    2 \times 3 &= 6 \mod 17\\
    6 \times 3 &= 1 \mod 17\\
\end{align*}
Notice that every element of $\mathbb{Z}_{17}^{\neq 0}$ appears in the list above. That is, $3$ is a generator for $\mathbb{Z}_{17}^{\neq 0}$.
\newline\newline
4. Using Fermat's theorem, find the remainder of $3^{47}$ when it is divided by 23.
\newline\newline
Notice that $3$ is prime and $23$ is prime, so clearly, $\gcd{(3, 23)} = 1$, so Fermat's theorem applies. Now, notice $3^{47} = 3^3(3^{22})^2$. By Fermat's theorem, we have $3^{22} \equiv 1 \mod 23$, so we have $3^3(3^{22})^2 \equiv 3^3(1)^2 \equiv 3^3 \equiv 4 \mod 23$.
\newline
That is, 
\[3^{47} \equiv 4 \mod 23\]
\newline\newline
10. Use Euler's generalization of Fermat's theorem to find the remainder of $7^{1000}$ when divided by 24.
\newline\newline
Begin by noticing that $\gcd{(7,24)} = 1$, so Euler's Generalization of Fermat's theorem applies, hereafter, Euler's theorem. By Euler's theorem, we have $7^{\phi(24)} \equiv 1 \mod 24$. From problem 7 (not shown), we have $\phi(24) = 8$, so $7^8 \equiv 1 \mod 24$. Now notice
\begin{align*}
    7^{1000} &= (7^{8})^{125} \\
    (7^8)^{125} &\equiv 1^{125} = 1 \mod 24 \\
\end{align*}
That is, 
\[7^{1000} \equiv 1 \mod 24\]
\newline\newline
\section*{Section 22 Problems}
5. How many polynomials are there of degree $\leq 3$ in $\mathbb{Z}_2[x]$? (Include 0.)
\newline\newline
I claim that there are $2^{3+1} = 2^4 = 16$ polynomials of degree $\leq 3$ in $\mathbb{Z}_2[x]$. To see this, observe the following list of polynomials:
\begin{align*}
    &0,\:\: 1 \\
    &x,\:\: 1 + x \\
    &x^2, \:\: x + x^2, \:\: 1 + x + x^2, \:\: 1 + x^2 \\
    &x^3,\:\: x^2 + x^3, \:\: x + x^2 + x^3, \:\: 1 + x + x^2 + x^3 \\
    &1 + x^3, \:\: 1 + x + x^3, \:\: x + x^3, \:\: 1 + x^2 + x^3 \\
\end{align*}
Which contains 16 polynomials.
\newline\newline
21. Consider the evaluation homomorphism $\phi_5 \: : \: \mathbb{Q}[x] \to \mathbb{R}$. Find six elements in the kernel of the homomorphism $\phi_5$.
\newline\newline
Notice that the following polynomials in $\mathbb{Q}[x]$ are in the kernel of $\phi_5$:
\begin{align*}
    f(x) &= x-5 \\
    g(x) &= x^2 - x - 20 \\
    h(x) &= x^3 - x^2 - x - 95 \\
    p(x) &= -4x^2 + 18x + 10 \\
    q(x) &= -\frac{90443}{30}x^3 + \frac{107014}{5}x^2 - \frac{970351}{30}x + 3501 \\
\end{align*}
and finally,
\newpage
\begin{align*}
    l(x) &= \frac{33949154613095804001}{100 000 000 000 000}x^7 - \frac{10185684021701526199}{1 250 000 000 000}x^6 + \frac{15347960110416856561}{200 000 000 000}x^5 - \frac{4482664476519146389}{12 500 000 000}x^4 + \cdots \\
    \\
    &\cdots \frac{21610109452604385023}{25 000 000 000}x^3 - \frac{2005784801822241183}{2 000 000 000}x^2 + \frac{21413820902381145861}{50 000 000 000}x + 1000\\
\end{align*}

27. Let $F$ be a field of characteristic zero and let $D$ be the formal polynomial differentiation map, so that 
\[D(a_0 + a_1x + a_2x^2 + \cdots + a_nx^n) = a_1 + 2\cdot a_2x + \cdots + n\cdot a_nx^{n-1}.\]
\begin{itemize}
    \item[\textbf{a.}] Show that $D : F[x] \to F[x]$ is a group homomorphism of $\langle F[x], + \rangle$ into itself. Is $D$ a ring homomorphism?
    \newline
    Proof: We must show that for $f(x), g(x) \in F[x]$, $D(f(x) + g(x)) = D(f(x)) + D(g(x))$. Well, let $f(x) \in F[x]$ be defined as $f(x) = a_0 + a_1x + \cdots + a_nx^n$ and similarly for $g(x) \in F[x]$, $g(x) = b_0 + b_1x + \cdots + b_nx^n$ where $a_i,b_i \in F$ for all $0 \leq i \leq n$. Begin by considering $D(f(x) + g(x))$:
    \newline
    By definition, we have
    \[f(x) + g(x) = (a_0 + b_0) + (a_1 + b_1)x + (a_2 + b_2)x^2 + \cdots + (a_n + b_n)x^n\]
    and so
    \begin{align*}
        D(f(x) + g(x)) &= (a_1 + b_1) + 2(a_2 + b_2)x + \cdots + n(a_n + b_n)x^{n-1} \\
        &= a_1 + 2a_2x + \cdots + na_nx^{n-1} + b_1 + 2b_2x + \cdots +nb_nx^{n-1} \\
        &= D(f(x)) + D(g(x)) \\
    \end{align*}
    So $D$ is a group homomorphism into itself. 
    \newline\newline
    However, $D$ is not a ring homomorphism. To see this, we must show that $D(f(x)\cdot g(x)) \neq D(f(x))\cdot D(g(x))$. Well,
    \begin{align*}
        f(x)\cdot g(x) &= (a_0 + a_1x + a_2x^2 + \cdots a_nx^n)(b_0 + b_1x + b_2x^2 + \cdots b_nx^n) \\
        &= a_0b_0 + a_0b_1x + a_0b_2x^2 + \cdots + a_0b_nx^n + \cdots \\
        &  a_1b_0 + a_1b_1x + a_1b_2x^2 + \cdots + a_1b_nx^n + \cdots \\
        & \vdots\\
        & a_nb_0 + a_nb_1x + a_nb_2x^2 + \cdots a_nb_nx^n \\
    \end{align*}
    Then
    \begin{align*}
        D(f(x) \cdot g(x)) &= a_0b_1 + 2a_0b_2x + \cdots + na_0b_nx^{n-1} + \cdots\\
        &a_1b_1 + 2a_1b_2x + \cdots + na_1b_nx^{n-1} + \cdots \\
        & \vdots \\
        &a_nb_1 + 2a_nb_2x + \cdots + na_nb_nx^{n-1} \\
    \end{align*}
    Not let us inspect $D(f(x))D(g(x))$:
    \begin{align*}
        D(f(x))D(g(x)) &= (a_1 + 2a_2x + \cdots + na_nx^{n-1})(b_1 + 2b_2x + \cdots nb_nx^{n-1}) \\
        &= a_1b_1 + 2a_1b_2x + \cdots na_1b_n + \cdots \\
        & 2a_2b_1x + 4a_2b_2x^2 + \cdots + 2na_2b_nx^n + \cdots\\
        & \vdots \\
        & na_nb_1x^{n-1} + 2na_nb_2x + \cdots + n^2a_nb_nx^{2n-2} \\
        &\neq D(f(x)g(x)) \\
    \end{align*}
    So $D$ is not a ring homomorphism.
    \newline
    
    \item[\textbf{b.}] Find the kernel of $D$.
    \newline
    Clearly, $f(x) = a \in \text{Ker}(D)$ for all $a \in F$. Additionally, since $F$ is a field, we have $F$ is an integral domain, so any polynomial of degree $\geq 1$ is not a zero divisor, so $\text{ker}(D) = \{f(x) = a \: | \: f(x) \in F[x], a \in F\} = F$.
    \newline
    
    \item[\textbf{c.}] Find the image of $F[x]$ under $D$.
    \newline
    Clearly, we have $\text{Im}(F[x]) = F[x]$ since for any $f(x) \in F[x]$, we can find a $g(x) \in F[x]$ such that $D(g(x)) = f(x)$. In fact, if $f(x) = a_0 + a_1x + \cdots + a_nx^n$, $g(x) = c + a_0x + a_1/2x^2 + \cdots + a_n/n!x^{n+1}$ where $c \in F$.
    
\end{itemize}


\end{document}
