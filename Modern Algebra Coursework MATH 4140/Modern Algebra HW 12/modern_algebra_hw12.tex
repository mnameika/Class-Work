\documentclass{article}
\usepackage[utf8]{inputenc}
\usepackage{amsmath}
\usepackage{amssymb}
\usepackage{mathtools}

\setlength{\oddsidemargin}{0in}
\setlength{\textwidth}{6.5in}
\setlength{\topmargin}{-.55in}
\setlength{\textheight}{9in}
\pagestyle{empty}


\title{Modern Algebra HW 12}
\author{Michael Nameika}
\date{November 2022}

\begin{document}

\maketitle

\section*{Section 27 Problems}
18. Is $\mathbb{Q}[x]/\langle x^2 - 5x + 6 \rangle$ a field? Why?
\newline\newline
$\mathbb{Q}[x]/\langle x^2 - 5x + 6\rangle$ is not a field because we may factor $x^2 - 5x + 6$ as $(x - 3)(x - 2)$ which shows us that $x^2 - 5x + 6$ is reducible in $\mathbb{Q}[x]$, and thus, $\mathbb{Q}[x]/\langle x^2 - 5x + 6 \rangle$ is not a field.
\newline\newline
19. Is $\mathbb{Q}[x]/\langle x^2 - 6x + 6 \rangle$ a field? Why?
\newline\newline
$\mathbb{Q}[x]/\langle x^2 - 6x + 6\rangle$ is a field. By the Eisenstein principle, let $p = 3$ and notice that 3 does not divide $a_2 = 1$ and 3 divides $a_1 = -6$ and $a_0 = 6$ but $3^2 = 9$ does not divide $a_0 = 6$. Thus, by the Eisenstein principle, $(x^2 - 6x + 6)$ is irreducible in $\mathbb{Q}$ and so $\mathbb{Q}[x]/\langle x^2 - 6x + 6 \rangle$ is a field.


\section*{Section 29 Problems}
18. a. Show that the polynomial $x^2 + 1$ is irreducible in $\mathbb{Z}_3[x]$.
\newline\newline
By the low degree test, it suffices to show that any element of $\mathbb{Z}_3$ is not a zero for $x^2 + 1$ in $\mathbb{Z}_3[x]$. Denote $f(x) = x^2 + 1$ and notice the following:
\begin{align*}
    f(0) &= 0^2 + 1 = 1 \\
    f(1) &= 1^2 + 1 = 1 + 1 = 2 \\
    f(2) &= 2^2 + 1 = 1 + 1 = 2 \\
\end{align*}
So $x^2 + 1$ is irreducible in $\mathbb{Z}_3[x]$.
\newline
b. Let $\alpha$ be a zero of $x^2 + 1$ in an extension field of $\mathbb{Z}_3$. As in Example 29.19, give the multiplication and addition tables for the nine elements of $\mathbb{Z}_3(\alpha)$, written in the order 0,1 2, $\alpha$, $2\alpha$, $1 + \alpha$, $1 + 2\alpha$, $2 + \alpha$, and $2 + 2\alpha$.
\newline\newline
Quickly note that in $\mathbb{Z}_3(\alpha)$, $\alpha^2 = \overline{-1}$. Now let us inspect the addition and multiplication tables.
\newline\newline
\begin{center}
    \begin{tabular}{c||c|c|c|c|c|c|c|c|c}
        + & 0 & 1 & 2 & $\alpha$ & $2\alpha$ & $1 + \alpha$ & $1 + 2\alpha$ & $2 + \alpha$ & $2 + 2\alpha$ \\
        \hline\hline
        0 & 0 & 1 & 2 & $\alpha$ & $2\alpha$ & $1 + \alpha$ & $1 + 2\alpha$ & $2 + \alpha$ & $2 + 2\alpha$ \\
        \hline
        1 & 1 & 2 & 0 & $1 + \alpha$ & $1 + 2\alpha$ & $2 + \alpha$ & $2 + 2\alpha$ & $\alpha$ & $2\alpha$ \\
        \hline
        2 & 2 & 0 & 1 & $2 + \alpha$ & $2 + 2\alpha$ & $\alpha$ & $2\alpha$ & $1 + \alpha$ & $1 + 2\alpha$ \\
        \hline
        $\alpha$ & $\alpha$ & $1 + \alpha$ & $2 + \alpha$ & $2\alpha$ & 0 & $1 + 2\alpha$ & 1 & $2 + 2\alpha$ & 2 \\
        \hline
        $2\alpha$ & $2\alpha$ & $1 + 2\alpha$ & $2 + 2\alpha$ & 0 & $\alpha$ & 1 & $1 + \alpha$ & 2 & $2 + \alpha$ \\
        \hline
        $1 + \alpha$ & $1 + \alpha$ & $2 + \alpha$ & $\alpha$ & $1 + 2\alpha$ & 1 & $2 + 2\alpha$ & 2 & $2\alpha$ & 0 \\
        \hline
        $1 + 2\alpha$ & $1 + 2\alpha$ & $2 + 2\alpha$ & $2\alpha$ & 1 & $1 + \alpha$ & 2 & $2 + \alpha$ & 0 & $\alpha$ \\
        \hline
        $2 + \alpha$ & $2 + \alpha$ & $\alpha$ & $1 + \alpha$ & $2 + 2\alpha$ & 2 & $2\alpha$ & 0 & $1 + 2\alpha$ & 1 \\
        \hline
        $2 + 2\alpha$ & $2 + 2\alpha$ & $2\alpha$ & $1 + 2\alpha$ & 2 & $2 + \alpha$ & 0 & $\alpha$ & 1 & $1 + \alpha$ \\
    \end{tabular}

    \begin{tabular}{c||c|c|c|c|c|c|c|c|c}
        $\cdot$ & 0 & 1 & 2 & $\alpha$ & $2\alpha$ & $1 + \alpha$ & $1 + 2\alpha$ & $2 + \alpha$ & $2 + 2\alpha$ \\
        \hline\hline
        0 & 0 & 0 & 0 & 0 & 0 & 0 & 0 & 0 & 0 \\
        \hline
        1 & 0 & 1 & 2 & $\alpha$ & $2\alpha$ & $1 + \alpha$ & $1 + 2\alpha$ & $2 + \alpha$ & $2 + 2\alpha$ \\
        \hline
        2 & 0 & 2 & 1 & $2\alpha$ & $\alpha$ & $2 + 2\alpha$ & $2 + \alpha$ & $1 + 2\alpha$ & $ 1 + \alpha$ \\
        \hline
        $\alpha$ & 0 & $\alpha$ & $2\alpha$ & 2 & 1 & $2 + \alpha$ & $1 + \alpha$ & $2 + 2\alpha$ & $1 + 2\alpha$ \\
        \hline
        $2\alpha$ & 0 & $2\alpha$ & $\alpha$ & 1 & 2 & $1 + 2\alpha$ & $2 + 2\alpha$ & $1 + \alpha$ & $2 + \alpha$ \\
        \hline
        $1 + \alpha$ & 0 & $1 + \alpha$ & $2 + 2\alpha$ & $2 + \alpha$ & $1 + 2\alpha$ & $2\alpha$ & 2 & 1 & $\alpha$ \\
        \hline
        $1 + 2\alpha$ & 0 & $1 + 2\alpha$ & $2 + \alpha$ & $1 + \alpha$ & $2 + 2\alpha$ & 2 & $\alpha$ & $2\alpha$ & 1 \\
        \hline
        $2 + \alpha$ & 0 & $2 + \alpha$ & $1 + 2\alpha$ & $2 + 2\alpha$ & $1 + \alpha$ & 1 & $2\alpha$ & $\alpha$ & 2 \\
        \hline
        $2 + 2\alpha$ & 0 & $2 + 2\alpha$ & $1 + \alpha$ & $1 + 2\alpha$ & $2 + \alpha$ & $\alpha$ & 1 & 2 & $2\alpha$ \\
        
        
        
    \end{tabular}
\end{center}


\end{document}
