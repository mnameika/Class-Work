\documentclass{article}
\usepackage[utf8]{inputenc}
\usepackage{amsmath}
\usepackage{amssymb}
\usepackage{mathtools}

\setlength{\oddsidemargin}{0in}
\setlength{\textwidth}{6.5in}
\setlength{\topmargin}{-.55in}
\setlength{\textheight}{9in}
\pagestyle{empty}


\title{Modern Algebra HW 8}
\author{Michael Nameika}
\date{November 2022}

\begin{document}

\maketitle

\section*{Section 18 Problems}
Decide whether the indicated operations of addition and multiplication are defined (closed) on the set, and give a ring structure. 
\newline\newline
\textbf{Problem 12} $\{a + b\sqrt{2} \:\: | \:\: a, b \in \mathbb{Q}\}$ with the usual addition and multiplication.
\newline
Let $S = \{a + b\sqrt{2} \:\: | \:\:  a, b \in \mathbb{Q}\}$. I claim that $\langle S, +, \cdot \rangle$ forms a field. 
\newline
Proof: To begin, we must show that $\langle S, + \rangle$ is an abelian group. To do so, we must first show $S$ is closed under addition. Well, let $\alpha, \beta \in S$ be defined as $\alpha = a_1 + b_1\sqrt{2}$ and $\beta = a_2 + b_2\sqrt{2}$ for the remainder of the proof and consider $\alpha + \beta$:
\begin{align*}
    \alpha + \beta &= a_1 + b_1\sqrt{2} + a_2 + b_2\sqrt{2} \\
    &= (a_1 + a_2) + (b_1 + b_2)\sqrt{2} \\
\end{align*}
By closure of $\mathbb{Q}$, we have $a_1 + a_2 \in \mathbb{Q}$ and $b_1 + b_2 \in \mathbb{Q}$, so $\alpha + \beta \in S$. Now, by associativity of addition in $\mathbb{Q}$, we have that $U$ is associative under $+$. Now, we must show that there exists an identity element. I claim that $e_+ = 0 + 0\sqrt{2}$ is the identity element of $S$ under addition. Consider $e_+ + \alpha$:
\begin{align*}
    e_+ + \alpha &= 0 + 0\sqrt{2} + a_1 + b_1\sqrt{2} \\
    &= a_1 + b_1\sqrt{2} \\
    &= a_1 + b_1\sqrt{2} + 0 + 0\sqrt{2} \\
    &= \alpha \\
\end{align*}
Now we must show that every element $\alpha \in U$ there exists some $\alpha^{-1} \in S$ such that $\alpha + \alpha^{-1} = \alpha^{-1} + \alpha = e_+$. I claim that for $\alpha$ defined above, we have $\alpha^{-1} = -a_1 - b_1\sqrt{2}$. Consider the following:
\begin{align*}
    \alpha + \alpha^{-1} &= a_1 + b_1\sqrt{2} + (-a_1 - b_1\sqrt{2}) \\
    &= a_1 - a_2 + (b_1 - b_2)\sqrt{2} \\
    &= 0 + 0\sqrt{2} = e_+ \\
    &= -a_2 - b_2\sqrt{2} + a_1 + b_1\sqrt{2} \\
    &= \alpha^{-1} + \alpha \\
\end{align*}
Thus, $\langle S, + \rangle$ forms a group. Additionally, by commutativity of addition in $\mathbb{Q}$, we have $\langle S, + \rangle$ is an abelian group. Now we must show that $\langle S^{\neq 0}, \cdot \rangle$ is a group. To begin, we must show that $S^{\neq 0}$ is closed under multiplication. Consider $\alpha \cdot \beta$:
\begin{align*}
    \alpha \cdot \beta &= (a_1 + b_1\sqrt{2})(a_2 + b_2\sqrt{2}) \\
    &= a_1a_2 + 2b_1b_2 + a_1b_2\sqrt{2} + a_2b_1\sqrt{2} \\
    &=(a_1a_2 + 2b_1b_2) + (a_1b_2 + a_2b_1)\sqrt{2} \\
\end{align*}
By closure of $\mathbb{Q}$, we have $a_1a_2 + 2b_1b_2 \in \mathbb{Q}$ and $a_1b_2 + a_2b_1 \in \mathbb{Q}$. Thus, $\alpha \cdot \beta \in S$, so $S$ is closed under multiplication. Now we must show that associativity holds for $S$. This holds by associativity of $\mathbb{Q}$. Now we must show that there exists an identity element in $S$.

I claim that $e = 1 + 0\sqrt{2} \in U$ is the identity. Well, let $\alpha = a_1 + b_1\sqrt{2} \in U$ and consider $e \cdot \alpha$:
\begin{align*}
    e \cdot \alpha &= (1 + 0\sqrt{2})\cdot (a_1 + b_1\sqrt{2}) \\
    &= a_1 + b_1\sqrt{2} \\
    &= (a_1 + b_1\sqrt{2})\cdot (1 + 0\sqrt{2}) \\
    &= \alpha \cdot e \\
    &= \alpha \\
\end{align*}

So $e$ is the identity for $\langle U, \cdot \rangle$. Now we must show that for every $\alpha \in U$, there exists $\alpha^{-1} \in U$ such that $\alpha^{-1}\cdot \alpha = \alpha \cdot \alpha^{-1} = e$. Let $\alpha = a_1 + b_1\sqrt{2} \in U$. I claim that $\alpha^{-1} = 1/(a_1 + b_1\sqrt{2})$. Well,
\begin{align*}
    \frac{1}{a_1 + b_1\sqrt{2}} &= \left(\frac{1}{a_1 + b_1\sqrt{2}}\right)\left(\frac{a_1 - b_1\sqrt{2}}{a_1 - b_1\sqrt{2}}\right) \\
    &= \frac{a_1 - b_1\sqrt{2}}{a_1^2 - 2b_1^2} \\
    &= \frac{a_1}{a_1^2 - 2b_1^2} - \frac{b_1}{a_1 - 2b_1^2}\sqrt{2} \\
\end{align*}
Now, for all $a_1, b_1 \in \mathbb{Q}$, we cannot have $a_1^2 = 2b_1^2$ (notice that this would give us that $\sqrt{2} \in \mathbb{Q}$ since we would get $a_1/b_1 = \sqrt{2}$), so $a_1/(a_1^2 - 2b_1), -b_1/(a_1^2 - 2b_1^2) \in \mathbb{Q}$, so $\alpha^{-1} \in S^{\neq 0}$. Thus, $\langle S^{\neq 0}, \cdot \rangle $ forms a group. Finally, we have established that $\langle S, +, \cdot \rangle$ forms a field.
\newline\newline
\textbf{37}. Show that if $U$ is the collection of all units in a ring $\langle R, +, \cdot \rangle $ with unity, then $\langle U, \cdot \rangle$ is a group. 
\newline\newline
Proof: Let $\langle U, \cdot \rangle$ be the collection of all units in a ring $\langle R, +, \cdot \langle $. We must show that $\langle U, \cdot \rangle$ is a group. To begin, we must show $\langle U, \cdot \rangle$ is closed under multiplication. Let $a, b \in U$ and consider $a \cdot b$. Let $c = a \cdot b$ and notice
\begin{align*}
    c \cdot b^{-1} &= a \cdot b \cdot b^{-1} \\
    c \cdot b^{-1} &= a \\
    c \cdot b^{-1} \cdot a^{-1} &= a \cdot a^{-1} \\
    c \cdot b^{-1} \cdot a^{-1} &= e \\
\end{align*}
where $e$ is the identity in $U$. That is, we have
\begin{align*}
    (a \cdot b)^{-1} &= b^{-1} \cdot a^{-1} \\
\end{align*}
and since $b^{-1}, a^{-1} \in R$, so $b^{-1} \cdot a^{-1} \in R$, and so $a \cdot b$ is a unit in $U$. Thus, $U$ is closed under multiplication. Now we must show that associativity holds. Since $U \subseteq R$, we have for any $a,b,c \in U$, $a,b,c \in R$, so associativity comes immediately from the fact that $\langle R, +, \cdot \rangle$ is a ring. Now we must show there exists an identity element. Well, since $R$ is a ring with unity, call the unity element $1_R$. Notice that $1_R \cdot 1_R = 1_R$, so $1_R$ is a unit in $R$. That is, $1_R \in U$. Thus, $1_R$ is the identity of $U$. Finally, we must show that for every $a \in U$. there exists an inverse element $a^{-1}$ such that $a \cdot a^{-1} = a^{-1} \cdot a = 1_R$. Well, since $U$ is a set of units in $R$, we get this for free! Thus, $\langle U, \cdot \rangle$ is a group.
\newline\newline
\textbf{41}. (Freshman exponentiation) Let $p$ be a prime. Show that in the ring $\mathbb{Z}_p$ we have $(a + b)^p = a^p + b^p$ for all $a,b \in \mathbb{Z}_p$. Also, is this result true in $\mathbb{Z}_6$?
\newline\newline
Proof: To begin, recall the binomial theorem:
\[(a + b)^p = \sum_{k=1}^n {n \choose k} a^{p-k}b^k\]
Expanding out a few terms, we see
\begin{align*}
    \sum_{k=1}^n {n \choose k} a^{p-k}b^k = {n \choose 0}a^p + {n \choose 1} a^{p-1}b + \cdots + {n \choose n-1}ab^{p-1} + {n \choose n}b^p \\
\end{align*}
Notice for all the terms ${n \choose l}$ contain a factor of $p$ in the numerator for $0 < l < n$, so ${n \choose l} = 0 \mod{p}$. That is, we are left with
\[\sum_{k=1}^n {n \choose k} a^{p-k}b^k = a^p + b^p\]
This result does not hold in $\mathbb{Z}_6$. Let $a = b = 2$ and first consider $(a + b)^6$:
\begin{align*}
    (a + b)^p &= (2 + 2)^6 \\
    &= 4^6 \\
    &= (2^3)^4 \\ 
    &= 2^4 \\
    &= 4\\
\end{align*}
Now consider $a^p + b^p$:
\begin{align*}
    a^p + b^p &= 2^6 + 2^6 \\
    &= (2^3)^2 + (2^3)^2 \\
    &= 2^2 + 2^2 \\
    &= 4 + 4 \\
    &= 2 \\
    &\neq (a + b)^p \\
\end{align*}

\section*{Section 19 Problems}
\textbf{3}. Find all solutions of the equation $x^2 + 3x + 2 = 0$ in $\mathbb{Z}_6$.
\newline\newline
Notice that we can factor $x^2 + 3x + 2$ as
\[x^2 + 3x + 2 = (x+2)(x+1)\]
so our "obvious" solutions are $x = -2 \equiv 4 \mod{6}$ and $x = -1 \equiv 5 \mod{6}$. Now, let us try the remaining elements of $\mathbb{Z}_6$ to see if they work:
\begin{align*}
    x&=1: \:\:\: (1+2)(1+1) = (3)(2) = 6 \equiv 0 \mod{6} \\
    x&=2: \:\:\: (2+2)(2+1) = (4)(3) = 12 \equiv 0 \mod{6} \\
    x&=3: \:\:\: (3+2)(3+1) = (5)(4) = 20 \equiv 2 \mod{6} \\
\end{align*}
So we can see the solutions to this equation in $\mathbb{Z}_6$ are $x = 1,2,4,5$.
\newline\newline
\textbf{23}. An element $a$ of a ring $R$ is \textbf{idempotent} if $a^2 = a$. Show that an integral domain contains exactly two idempotent elements.
\newline\newline
Proof: Let $R$ be an integral domain. We wish to show that $R$ has exactly two idempotent elements. To begin, the zero element of $R$ is idempotent:
\[0\cdot 0 = 0\]
Additionally, since $R$ is an integral domain, $R$ has a unity element, call it $1_R$, and $1_R$ is an idempotent:
\[1_R \cdot 1_R = 1_R\]
Now we must show that there exist no other idempotent elements. Suppose by way of contradiction that there exists another idempotent element $a \in R$, $a \neq 0, 1_R$. Then 
\[a^2 = a\]
but since $R$ is an integral domain, $a$ has a multiplicative inverse and cancellation holds, so 
\begin{align*}
    \frac{1}{a}a^2 &= \frac{1}{a}a \\
    a &= 1_R \\
\end{align*}
contradicting our assumption that $a \neq 1_R$. Thus, an integral domain contains exactly two idempotent elements.


\end{document}
