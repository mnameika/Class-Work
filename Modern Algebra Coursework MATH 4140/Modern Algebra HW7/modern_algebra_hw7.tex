\documentclass{article}
\usepackage[utf8]{inputenc}
\usepackage{amsmath}
\usepackage{amssymb}
\usepackage{mathtools}

\setlength{\oddsidemargin}{0in}
\setlength{\textwidth}{6.5in}
\setlength{\topmargin}{-.55in}
\setlength{\textheight}{9in}
\pagestyle{empty}

\title{Modern Algebra HW7}
\author{Michael Nameika}
\date{October 2022}

\begin{document}

\maketitle

\section*{Section 15 Problems}

For the following, (1) compute the order of each element in the factor group and (2) determine whether or not the factor group is cyclic.
\newline

\textbf{2.} $(\mathbb{Z}_2 \times \mathbb{Z}_4)/\langle (0,2) \rangle$
\newline
(1) Let us begin by computing $\langle (0,2) \rangle$:
\[\langle (0,2) \rangle = \{(0,0), (0,2)\}\]
So $\langle (0,2) \rangle$ has 2 elements, and since $\mathbb{Z}_2 \times \mathbb{Z}_4$ has 8 elements, $(\mathbb{Z}_2 \times \mathbb{Z}_4)/\langle (0,2) \rangle$ has 4 elements. Now let's find those elements:
\begin{align*}
    (0,0) + \langle (0,2) \rangle &= \{(0,0), (0,2)\} \\
    (1,0) + \langle (0,2) \rangle &= \{(1,0), (1,2)\} \\
    (1,1) + \langle (0,2) \rangle &= \{(1,1), (1,3)\} \\
    (0,1) + \langle (0,2) \rangle &= \{(0,1), (0,3)\} \\
\end{align*}
And now let us calculate the order of each element:
\begin{align*}
    \left| (0,0) + \langle (0,2) \rangle \right| &= 1 \\
    \left| (1,0) + \langle (0,2) \rangle \right| &= 2 \\
    \left| (1,1) + \langle (0,2) \rangle \right| &= 2 \\
    \left| (0,1) + \langle (0,2) \rangle \right| &= 2 \\
\end{align*}

(2) Since $(\mathbb{Z}_2 \times \mathbb{Z}_4) /\langle (0,2) \rangle$ has four elements and each element has order less than four, this factor group cannot be cyclic!
\newline\newline

\textbf{3.} $(\mathbb{Z}_2 \times \mathbb{Z}_4)/\langle (1,2) \rangle$
\newline
(1) Let us begin by computing $\langle (1,2) \rangle$:
\[\langle (1,2) \rangle = \{(0,0), (1,2)\}\]
Similar to problem 2, we have that $\langle (1,2) \rangle$ will have four elements. Let's list them:
\begin{align*}
    (0,0) + \langle (1,2) \rangle &= \{(0,0), (1,2)\} \\
    (0,1) + \langle (1,2) \rangle &= \{(0,1), (1,3)\} \\
    (1,0) + \langle (1,2) \rangle &= \{(1,0), (0,2)\} \\
    (1,1) + \langle (1,2) \rangle &= \{(1,1), (0,3)\} \\
\end{align*}
And now let us calculate the order of each element:
\begin{align*}
    \left| (0,0) + \langle (1,2) \rangle \right| &= 1 \\
    \left| (0,1) + \langle (1,2) \rangle \right| &= 4 \\
    \left| (1,0) + \langle (1,2) \rangle \right| &= 2 \\
    \left| (1,1) + \langle (1,2) \rangle \right| &= 4 \\
\end{align*}
(2) Since $(\mathbb{Z}_2 \times \mathbb{Z}_4)/\langle (1,2) \rangle$ has four elements and two elements have order 4, we have that $(\mathbb{Z}_2 \times \mathbb{Z}_4)/\langle (1,2) \rangle$ is a cyclic group.

\section*{Bonus Problem!!!}

Prove if $G$ is a cyclic group and $H$ is any subgroup of $G$ then $G/H$ is cyclic.
\newline\newline
Proof: Let $G$ be a cyclic group and $H \leq G$. We wish to show that $G/H$ is cyclic. Well, by definition, the elements of $G/H$ are cosets of $H$. Additionally, since $G$ is cyclic, there exists some $a \in G$ such that $\langle a \rangle = G$. Now, I claim that $aH$ generates $G/H$. Well, repeatedly composing $aH$ with itself $n$ times gives $a^nH$, and since $a$ generates $G$, we will eventually find every coset of $H$. So $aH$ generates $G/H$, so $G/H$ is cyclic.  

\end{document}
